%\section{Implementation}
%\label{sec:impl}
%\subsection{Implementation}
We have implemented the %\aivcalg
algorithm
in an industrial model checker called \texttt{JKind} ~\cite{jkind},
which verifies safety properties of  infinite-state synchronous systems.
It accepts Lustre programs \cite{Halbwachs91:lustre} as input.  The translation of Lustre
into a symbolic transition system in \texttt{JKind} is straightforward and is similar to what is described
in~\cite{Hagen08:FMCAD}.
Verification is supported by multiple ``proof engines'' that execute in parallel, including K-induction,
property directed reachability (PDR), and lemma generation.  During verification,
\texttt{JKind} emits SMT problems using the theories of linear integer and real arithmetic, and can use the
%I remove citations to save some space in the end
%------------------
%\texttt{Z3}~\cite{DeMoura08:z3},
%\texttt{Yices}~\cite{Dutertre06:yices}, \texttt{MathSAT}~\cite{Cimatti2013:MathSAT},
%\texttt{SMTInterpol}~\cite{Christ2012:SMTInterpol}, and \texttt{CVC4} \cite{barrett2011cvc4}
%----------------------------------------------------------------
\texttt{Z3}, \texttt{Yices}, \texttt{MathSAT}, \texttt{SMTInterpol}, and \texttt{CVC4} SMT solvers as back-ends.  When a property is
proved and IVC generation is enabled, an additional parallel engine
executes the \ucalg ~algorithm \cite{Ghass16} to generate an (approximately) minimal IVC.
%
To implement our method, we have extended \texttt{JKind} with a new engine that
implements Algorithm \ref{alg:allmivc} on top of \texttt{Z3}.
%The source code is publicly available on \cite{mygit}.
We use the \texttt{JKind} IVC generation engine to implement the \ucalg\ procedure in  Algorithm \ref{alg:allmivc}.
%It worth mentioning that we could have employed the \ucbfalg ~algorithm for this purpose as well.
%One might say that since \ucbfalg ~guarantees minimality, it would help to the \aivcalg ~algorithm terminate more quickly.
%However, as we will show in the experiments, on the one hand, the \ucbfalg ~algorithm is very expensive. On the other hand, the \ucalg ~algorithm is not only fast, but it also generates \ivc s that are
%quite close to the \mivc s computed by \ucbfalg.

%\todo{ALL: how do timeouts fit into the general framework?}

\iffalse
As mentioned in Section~\ref{sec:allivcs} the \isadeq\ procedure may not terminate.
In our implementation, we measure the time required to prove the property and the initial
given the full model (\emph{proof-time}), and the time required to calculate the first
(approximate) IVC using \ucalg\ (\emph{\ucalg-time}).
We then set a timeout for each iteration of the \aivcalg\ algorithm to
($30$ sec  $+\ 5\ \times$ (\emph{proof-time} $+$ \emph{\ucalg-time})).
In almost all cases in our experiment and our use of the tools,
this timeout is sufficient to ensure exact results.
%In the experiment, only 15 of 475
%models (3\%) had potentially approximate results.
%Since the timeout is used by both the brute-force
%algorithm and the \aivcalg\ algorithm, minimality is only guaranteed if there
%are no timeouts.  %In addition, because the analysis problems are slightly different
%between the two approaches, timeouts can occasionally lead to anomalies where the
%brute force algorithm discovers a {\em smaller} MIVC than is discovered by the
%\aivcalg ~procedure.  %This anomaly occurs in one of the benchmark examples within our
%experiment.
It is important to note that by increasing the timeout, it is possible that
in some cases smaller IVCs can be generated, but the general problem will remain due
to the undecidability of the model checking problem.
\fi



%The source code is publicly available on \cite{mygit}.
%This section points out some technical issues in the implementation.
%
%We have extended \texttt{JKind} with a new engine that
% implements Algorithm \ref{alg:aivc}.
% First a property is proved and a single IVC set $S$ is computed
% by the IVC generation engine. Then, the set $S$ is sent to our engine.
% For efficiency, instead of initializing $map$ with $\top$,
% we initialize $map$ with $\bigvee_{T_{i}\in S} \neg {\actlit (T_i)}$,
% which prunes off set $S$ and all its supersets from $\mathcal{P}(T)$ at the beginning of the algorithm. If $S$ is minimal, then chances are a large
% subset of  $\mathcal{P}(T)$ is marked as explored even before
% the algorithm starts the search.

