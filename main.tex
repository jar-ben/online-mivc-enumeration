\documentclass{llncs}

\begin{document}

\title{Online Enumeration of All Minimal Inductive Validity Cores}

\author{Jaroslav Bend\'ik\inst{1} 
	\and Elaheh Ghassabani\inst{2}
	\and Michael Whalen\inst{2}
	\and Ivana \v Cern\'a\inst{1}
}

\institute{Faculty of Informatics, Masaryk University, Brno, Czech Republic\\
\email{\{xbendik,cerna\}@fi.muni.cz}
\and
Department of Computer Science \& Engineering, University of Minnesota, MN, USA\\
\email{\{ghass013,mwwhalen\}@umn.edu}
}


\maketitle    
\begin{abstract} 
Symbolic model checkers can construct proofs of
safety properties over complex models, but when a proof succeeds,
the results do not generally provide much insight to
the user. For explanation of the correctness, Minimal Inductive Validity Cores (MIVCs) can be used. Every MIVC traces the property to a minimal set of model elements necessary
for constructing a proof. 
The traceability information provided by MIVCs can be used to perform a variety of engineering analysis such as a coverage analysis, a robustness analysis or a vacuity detection. 
The more MIVCs are identified, the more precise analyses can be performed. 
However, a full enumeration of all MIVCs
is in general intractable due to the combinatorial explosion.

The bottleneck of existing algorithms is that they identify all MIVCs simultaneously at the very end of the computation.
Thus they output no MIVC in the cases where the complete enumeration is intractable. 
In this paper, we propose an algorithm that identifies MIVCs in an \emph{online} manner (i.e., one by one)
and can be terminated anytime.
We benchmark our algorithm against existing algorithms on a variety of benchmarks. We show that our algorithm not only is the winner in the intractable cases but also completes the enumeration sooner in many tractable cases.


\keywords{Inductive Validity Cores, SMT-based model checking, Inductive proofs, Traceability}
\end{abstract} 
 
 
 
\section{Introduction}
\label{sec:intro}
Symbolic model checking using induction-based techniques such as IC3/PDR~\cite{Een2011:PDR}, $k$-induction~\cite{SheeranSS00}, and $k$-liveness~\cite{conf/fmcad/ClaessenS12} can be used to determine whether properties hold of complex finite or infinite-state systems.  Such tools are popular both because they are highly automated (often requiring no user interaction other than the specification of the model and desired properties), and also because, in the event of a violation, the tool provides a counterexample demonstrating a situation in which the property fails to hold.  These counterexamples can be used both to illustrate subtle errors in complex hardware and software designs~\cite{hilt2013,Miller10:CACM} and to support automated test case generation~\cite{Whalen13:OMCDC,You15:dse}.

If a property is proved, however, most model checking tools do not provide additional information.  This can lead to situations in which developers have an unwarranted level of confidence in the behavior of the system.  Issues such as vacuity~\cite{Kupferman03:Vacuity}, incorrect environmental assumptions~\cite{Whalen07:FMICS}, and errors either in English language requirements or formalization
%~\cite{Pike06:axioms} 
can all lead to failures of ``proved'' systems.  Thus, even if proofs are established, one must approach verification with skepticism.

Recently, {\em proof cores}~\cite{jasper_gold} have been proposed as a mechanism to determine which elements of a model are used when constructing a proof.  This idea is formalized by Ghassabani et al. for inductive model checkers~\cite{Ghass16} as {\em Inductive Validity Cores} (IVCs). IVCs offer proof explanation as to why a property is satisfied by a model in a formal and human-understandable way.  The idea lifts UNSAT cores~\cite{zhang2003extracting}
to the level of sequential model checking algorithms using induction.  Informally, if a model is viewed as a conjunction of constraints,
a minimal IVC (MIVC) is a set of constraints that is sufficient to construct a proof such that if any constraint is removed, the property is no longer valid.
%
Depending on the model and property to be analyzed, there are many possible MIVCs, and there is often substantial diversity between the IVCs used for proof.
%
In previous work~\cite{Ghass16,Murugesan16:renext,Ghass17Cov,Ghass17AllIVCs} we have explored several different uses of IVCs, including:

\noindent \textbf{Traceability: } %For functional properties that can be proven with inductive model checkers, 
Inductive validity cores can provide accurate traceability matrices with no user effort.  Given multiple IVCs, {\em rich traceability} matrices~\cite{Murugesan16:renext} can be automatically constructed that provide additional insight about {\em required} vs. {\em optional} design elements.

\noindent \textbf{Vacuity detection:} Syntactic vacuity detection (checking whether all subformulae within a property are necessary for its validity) has been well studied~\cite{Kupferman03:Vacuity}.   IVCs allow a generalized notion of vacuity that can indicate weak or mis-specified properties even when a property is syntactically non-vacuous.

\noindent \textbf{Coverage analysis:} Coverage analysis provides a metric as to whether a set of properties is adequate for the model.  Several different notions of coverage have been proposed~\cite{chockler_coverage_2003,kupferman_theory_2008}, but these tend to be very expensive to compute.  IVCs provide an inexpensive coverage metric by determining the percentage of model atoms necessary for proofs of all properties.

\noindent \textbf{Impact Analysis:} Given a single (or for more accurate results, all) MIVCs, it is possible to determine which requirements may be falsified by changes to the model.  This analysis allows for selective regression verification of tests and proofs: if there are alternate proof paths that do not require the modified portions of the model, then the requirement does not need to be re-verified.

\noindent \textbf{Design Optimization:} A practical way of calculating all MIVCs allows synthesis tools to find a minimum set of design elements (optimal implementation) for a certain behavior. Such optimizations can be performed at different levels of synthesis.

To be useful for these tasks, the generation process must be efficient and the generated IVC must be accurate and precise (that is, sound and minimal).  In previous work, we have developed an efficient {\em offline} algorithm~\cite{Ghass17AllIVCs} for finding all minimal IVCs based on the MARCO algorithm for MUSes~\cite{marco2016fast}.  The algorithm is considered {\em offline} because it is not until all IVCs have been computed that one knows whether the solutions computed are, in fact, minimal.  In cases in which models contain many IVCs, this approach can be impractically expensive or simply not terminate.

In this paper, we propose a novel {\em online} algorithm for MIVC enumeration.  With this algorithm, solutions are produced incrementally, and each solution produced is guaranteed to be minimal.
Therefore, the algorithm produces at least some MIVCs even in the case of models for which is a complete MIVC enumeration intractable. Moreover, the proposed algorithm is often more efficient then the baseline MARCO also in the case of tractable models.
%Additionally, for models with a large number of IVCs, the proposed algorithm is considerably more efficient than the baseline MARCO algorithm.
We demonstrate this via an experimental evaluation.

The rest of the paper is organized as follows. In Section~\ref{sec:preliminaries} we define all the necessary notions. Section~\ref{sec:existing-techniques} summarizes the existing techniques. In Section~\ref{sec:algorithm} we present our novel algorithm. Section~\ref{sec:example-execution} provides an example execution of our algorithm. Finally, sections \ref{sec:implementation} and \ref{sec:experiment} cover implementation details and present experimental results. 	




\section{Motivating Example}
\label{sec:mot-example}
\input{motivation}


\section{Preliminaries}
\label{sec:preliminaries}
A transition system $(I,T)$ over a state space $S$ consists of an initial state predicate $I : S \rightarrow bool$ and a transition step predicate $T : S \times S \rightarrow bool$. The notion of reachability for $(I, T)$ is defined as the smallest predicate $R : S \rightarrow bool$ satisfying the following formulae:

\begin{itemize}
	\item[] $\forall s \in S: \, I(s) \Rightarrow R(s)$
	\item[] $\forall s, s' \in S: \, R(s) \wedge T(s, s') \Rightarrow R(s')$
\end{itemize}

A safety property $P: S \rightarrow bool$ holds on a transition system $(I, T)$ iff it holds on all reachable states, i.e., $\forall s \in S: \, R(s) \Rightarrow P(s)$. We denote this by $(I, T) \vdash P$. We assume
the transiton step predicate $T$ is equivalent to   a conjunction of transition step predicates $T_1, \ldots, T_n$,  called top level conjuncts. 
%the transition relation has the structure of a top-level conjunction $T(s, s') = T_1(s, s') \wedge \cdots \wedge T_n(s, s')$. 
In such case, $T$ can be identified with the set of its top level conjuncts $\{ T_1, \ldots, T_n\}$. By further abuse of notation, we write $T \setminus \{ T_i \}$ to denote removal of top level conjunct $T_i$ from $T$, and $T \cup \{ T_j\}$ to denote addition of top level conjunct $T_j$ to $T$. 


\begin{definition}
A set of conjucnts $U \subseteq T$ is an Inductive Validity Core (IVC) for $(I, T) \vdash P$ iff $(I, U) \vdash P$. Moreover, $U$ is a Minimal IVC (MIVC) for $(I, T) \vdash P$ iff $(I, U) \vdash P$ and $\forall U_i \in U: \, (I, U \setminus \{ U_i\}) \nvdash P$.
\end{definition}

Note, that the minimality is with respect to the set inclusion and not wrt cardinality. There can be multiple MIVCs with different cardinalities. 

%\textit{concept used here is set minimality, not minimum cardinality. This means, that there can be multiple MIVCs with different cardinalities.} 

\todo{Example}







\section{Related Work}
\label{sec:related-work}
\input{related_work}


\section{Algorithm}
\label{sec:algorithm}
In this section, we propose a novel algorithm for online enumeration of MIVCs. The algorithm is build out of two basic procedures: shrink adn grow. The grow procedure is symmetric to the shrink one.\\
 

\subsection{Efficient Shrink Procedure}
Let us recall that the shrink procedure  turns an adequate subset into a MIVC. The problem with the shrinking procedure UC\_BF NOVY NAZOV presented in the previous section is, that it performs linearly many checks for adequacy with respect to the size of the given  adequate subset. Since each check for adequacy is in fact a  model checking problem, the shrink procedure can be thus extremely time demanding.   
We can improve  the complexity of shrinking if we use the fact  that we are not searching just for   one MIVC but for all MIVCs. Therefore, we can use information obtained during the previous computation to speed up the future shrinks. In particular, once we determine the status of $U \subseteq T$, i.e. once $U$ become explored, we can avoid checking all of its subsets or supersets, respectively, for adequacy during subsequent shrink procedures. 
Algorithm~\ref{alg:shrink_efficient} shows an improved shrinking procedure. Similarly to $UC\_BF$ algorithm, it iteratively attempts to remove elements from the set that is being
shrunk, checking each new set for adequacy and keeping only changes that
leave the set adequate. However, it avoids checking explored subsets for adequacy since their status is already known.

\begin{algorithm}[!t]
%\documentclass[]{article}
%\usepackage[]{algorithm2e}
%
%\begin{document}
%\begin{algorithm}

\SetKwInOut{Input}{input}\SetKwInOut{Output}{output}

\DontPrintSemicolon

\Input{$(I, T) \vdash P$}
\Input{set $Unex$ of unexplored subsets of $T$}
\Output{MIVC for $(I, T) \vdash P$}
	\For{$T_i \in T$}{
		\If{$ T \setminus \{ T_i \} \in Unex$ }{		
			\If{$(I, T \setminus \{ T_i \}) \vdash P$}{
				$T \gets T \setminus T_i$\;
			}
		}
	}
	\Return $T$\;

%\end{algorithm}




%\end{document}

\caption{Novel shrinking algorithm.}
\end{algorithm}

Intuitively, the more subsets are   explored, the more checks for adequacy can be saved during the shrinking. The question is which subsets should be checked for adequacy (i.e., become explored) before we employ the shrinking procedure. The answer to this question can be found in the following:

\begin{theorem}
\label{theorem:unex-prop}
Let $U_1, U_2$ be subsets of $T$ such that $U_1$ is explored, $U_2$ is unexplored, and $U_1 \subset U_2$. Then $U_1$ is inadequate.
\end{theorem}

\begin{proof}
If $U_1$ is adequate, then all of its supersets are necessarily adequate. Thus, if $U_1$ is determined to be adequate, then not just $U_1$ but also all of its supersets becomes explored. Since $U_1$ is explored and $U_2$ is unexplored, then $U_1$ is necessarily an inadequate subset of $T$.
\end{proof}

\begin{corollary}
Let $U_1, U_2$ be subsets of $T$ such that $U_2$ is explored, $U_1$ is unexplored, and $U_1 \subset U_2$. Then $U_1$ is adequate.
\end{corollary}

\begin{proof}
Immediately from Theorem~\ref{theorem:unex-prop}. 
\end{proof}

Assuming that we employ the shrinking algorithm only on unexplored subsets, Theorem~\ref{theorem:unex-prop} implies that each unexplored subset that is encountered during the shrinking is necessarily an inadequate subset. Therefore, it is knowledge of adequacy of inadequate subsets that can be exploited to boost the shrinking procedure since we can skip some adequacy checks. One can say that in order to save a one check for adequacy during shrinking, we have to perform the same one check before the shrinking, i.e. no checked is saved at the end. But this is not true due to two reasons. First, since we are looking for multiple MIVCs, we might perform multiple shrinks, and thus we might encounter one particular inadequate subset several times. Second, if we find out that $U \subseteq T$ is inadequate, than we determine that all of its subsets are also inadequate, and thus, we can save checking all of these subsets for adequacy during subsequent shrinks. 

\begin{algorithm}[!t]
%\documentclass[]{article}
%\usepackage[]{algorithm2e}
%
%\begin{document}
%\begin{algorithm}

\SetKwInOut{Input}{input}\SetKwInOut{Output}{output}
\SetKwFunction{getMIVCApproximation}{getMIVCApproximation}


\DontPrintSemicolon

\Input{$(I, T) \vdash P$}
\Input{$U \subset T$ such that  $(I, U) \nvdash P$}
\Input{set $Unex$ of unexplored subsets of $T$}
\Output{approximately maximal inadequate subset of $\mathcal{T}$.}

	$M \gets $ a maximal $M \in Unex$ such that $M \supseteq U$\;
	\While{$(I, M) \vdash P$}{
			$\mathit{MIVCApprox} \gets \getMIVCApproximation((I,M), P)$\;
			$T_i \gets choose T_i \in (\mathit{MIVCApprox} \cap (T \setminus U))$\;
			$M \gets M \setminus \{ T_i \}$\;
	}
	
	\Return $M$\;
	
%\end{algorithm}




%\end{document}

\caption{Approximate grow algorithm.}
\end{algorithm}



\subsection{Efficient Grow Procedure}
In this section, we present an efficient way how to find inadequate subsets in order to boost the subsequent calls of the shrinking procedure.

Intuitively, in order to identify inadequate subsets as quickly as possible, we should strive for finding maximal inadequate subsets (MISes).  
In order to find an MIS, we can use a \emph{grow} procedure which turns and inadequate subsets into a MIS, i.e. it is a dual of the shrink procedure. However, such procedure might be quite expensive. Similarly as in the case of UC\_BF we can iteratively attempt to add elements to the set that is being
grown, checking each new set for adequacy and keeping only changes that
leave the set inadequate. However, such approach performs linearly many adequacy checks which is rather inefficient. We can boost this procedure in the same way as is boosted the shrinking procedure, i.e. exploit subsets that are already known to be adequate to save some adequacy checks. However, this would mean that we are using a grow procedures to speed shrinks procedures, but to speed the grow procedures we need shrink procedures. It is easy to see, that this is not the right way. Thus, we propose a different approach. 

Instead of finding MISes, we propose to search for approximately MISes. Algorithm~\ref{alg:grow-approx} shows an approach how can be an unexplored inadequate subset $U$ turned into an approximately MIS. It first finds some maximal unexplored superset $M$ such that $M \supseteq U$ and then checks it for adequacy. Note, that if a maximal unexplored subset is inadequate, then it is necessarily an MIS - this is a straightforward consequence of Corollary~\ref{corollary:unex-prop}. 
Otherwise, if $M$ is adequate then it is iteratively reduced until an inadequate subset is found. In each iteration, we first use algorithm IVC\_UC to find an approximately minimal IVC $M_{IVC}$ of $M$. $M_{IVC}$ is a succinct representation of the reason (or one of the reasons) of $M$ being adequate. Therefore, we choose some $T_i \in M_{IVC}$ and remove it from $M$ attempting to make it inadequate. We also require $T_i$ to be included in $T \setminus U$, i.e. in $U$'s complement. This guarantees that $M$ is a superset of $U$ during the whole computation and eventually an inadequate subset is found since $U$ is inadequate. Moreover, since both $U$ and $M$ were initially unexplored, this condition on $T_i$ also implies that the resultant $M$ was also unexplored at the start of the growing, i.e. an unexplored approximately maximal inadequate subset is found. 

\subsection{Complete Algorithm}
In this section, we describe, how to combine the shrink and grow methods in order to form an efficient online MIVC enumeration algorithm.

Since knowledge of (approximately) maximal inadequate subsets might be used to speed up shrinking procedures, it might be tempting to first find all maximal inadequate subsets. However, there can be up to exponentially many such subsets with respect to the size of $T$. Thus, finding first all maximal inadequate subsets is in general intractable. Instead, we propose to alternate both shrinking and growing procedures. Note, that during shrinking, we might determine some subsets to be inadequate, and such subsets can be subsequently used as \emph{seeds} for growing. Dually, we might determine some subsets to be adequate during the growing procedures, and such subsets can be used as \emph{seeds} for shrinking procedures. Thus, both these procedures somehow complements each other. 

Algorithm~\ref{alg:core} shows our algorithm for online enumeration of all MIVCs. It iteratively... todo popis algoritmu + pseudokod. 


\begin{algorithm}[!t]
\DontPrintSemicolon
\SetKwInOut{Input}{input}\SetKwInOut{Output}{output}
\SetKwFunction{Shrink}{Shrink}
\SetKwFunction{FindMIVCs}{FindMIVCs}
\SetKwFunction{Grow}{Grow}
\SetKwFunction{Dequeue}{Dequeue}
\SetKwProg{Fn}{Function}{:}{}

\Fn{\FindMIVCs{$(I, T) \vdash P$}}{
	$\mathit{Unex} \gets \mathcal{P}(T)$\;
	$\mathit{shrinkingQueue} \gets $ empty queue\;
	\While{$Unex \neq \emptyset$}{
		$B \gets$ a minimal $B \in Unex$\;
		\eIf{$(I, B) \vdash P$}{
			\textbf{output} $B$\;
			$\mathit{Unex} \gets \mathit{Unex} \setminus \mathit{Sup}(B)$\;
		}{
			$\mathit{MIS} \gets \Grow(B)$ \tcp*{side effect: fills $\mathit{shrinkingQueue}$}
			$\mathit{Unex} \gets \mathit{Unex} \setminus \mathit{Sub}(MIS)$\;			
		} 
		\While{$\mathit{shrinkingQueue}$ is not empty}{
			$\mathit{seed} \Dequeue(\mathit{shrinkingQueue})$\;
			$MIVC \gets \Shrink(seed)$\;
			remove from $\mathit{shrinkingQueue}$ each $V$ such that $V \supset MIVC$\;
			\textbf{output} $MIVC$\;
			$\mathit{Unex} \gets \mathit{Unex} \setminus \mathit{Sup}(MIVC)$\;
		}
	}
}

\setcounter{AlgoLine}{0}
\Fn{\Shrink{$U$}}{
%\Input{an unsatisfiable set of constraints $C$}
	\For{$T_i \in U$}{
		\If{$ U \setminus \{ T_i \} \in Unex$ }{		
			\If{$(I, U \setminus \{ T_i \}) \vdash P$}{
				$U \gets T \setminus T_i$\;
			}
		}
	}
	\Return $U$\;
}

\setcounter{AlgoLine}{0}
\Fn{\Grow{$B$}}{
	$M \gets $ a maximal $M \in Unex$ such that $M \supseteq B$\;
	\While{$(I, M) \vdash P$}{
			$\mathit{MIVCApprox} \gets \getMIVCApproximation((I,M), P)$\;
			$T_i \gets choose T_i \in (\mathit{MIVCApprox} \cap (T \setminus M))$\;
			$M \gets M \setminus \{ T_i \}$\;
	}	
	\Return $M$\;
}	 


\caption{Our novel algorithm for MIVC enumeration.}
\end{algorithm}




 


\section{Implementation}
\label{sec:impl}
\input{implementation}


\section{Experiment}
\label{sec:experiment}
\todo{Jaroslav: Briefly state research questions, and provide plots and scripts for plots}\\
\todo{Elaheh: finish experimental section (once I (Jaroslav) provide data)}



\end{document}